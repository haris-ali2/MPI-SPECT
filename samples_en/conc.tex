\chapter{Conclusion}
\label{ch:conclusion}

This research investigated the intersection of applications of \gls{dl} with prior-based anatomical modeling in the context of \gls{mpi} using \gls{spect}. Specifically, this research shows the strength of integrating the transformer architecture, who are known for their capacity to learn long-range dependencies, with \gls{ssp} which introduces a very powerful inductive bias that improves the accuracy of the segmentation, the robustness and the anatomical consistency. Through a very extensive set of experiments, both on the real-world patient data and also the synthetic phantoms the proposed hybrid model showed measurable improvements in all of the core metrics of segmentation when compared to the transformer baseline such as nnFormer and also the Swin-UNetR. Notably the proposed model excelled not only in high fidelity data settings but also in regimes where there is a limited amount of data available. Here, the \gls{ssp}s played a very important role in guiding the training process and to reduce the overfitting of the model. These results demonstrate the advantage of using structured prior knowledge into data driven models especially in domains like medical imaging where the annotated data is not available in abundance, the anatomical plausibility is critical.

The methodology used in this study also proved to be extremely adaptable functioning very affectively across a range of \gls{snr} levels and also nuder a diverse amount of noise. The use of shape prior based phantoms in order to test the robustness to noise validated the generalization capacity of the the model even further and also its resilience to common imperfections that occur in the workflows of clinical imaging. Moreover, the analysis of the learned representations using UMAP embeddings indicated more structured and discriminative latent space for the method proposed in this study, which suggests enhanced abstraction of features and a more in-depth understanding of the underlying distribution of anatomies. One of the major key finding is that even with a compact and modest dataset size of 50 to 60 labeled patients, the proposed model achieved a strong generalizational power to data that is unseen. This provides with an encouraging proof for future work in deploying such an architecture in real-world settings where te access to large amount of annotated data is most of the times impractical due to a lot of critical financial, ethical and operational limitations. The study confirms that reliable and standardized tools for segmentation of \gls{mpi} \gls{spect} are achievable without the necessity of a huge amount of data.

From the perspective of the clinics ,standardized and accurate segmentation can very significantly streamline the downstream tasks such as the functional quantification (such as \gls{lvef}, \gls{edv}, \gls{esv}), rish mitigation and therap planning. The incorporation of transformer-based segmentation pipelines, augmented with the prior knowledge, lays a foundation for improving the diagnostic reproducibility and enhancing the decision making process of the clinics. Future work can build upon this foundation by incorporating dynamic priors exploring semi-supervised techniques, and validating the present approach across a number of different institutions and imaging modalities.
