\chapter{Introduction}
\label{ch:intro}

\gls{mpi} using \gls{spect} pllays an important role in the process of non-invasive assessment of the \gls{cad}. Considering cardiovascular diseases being one of the leading causes of mortality all across the world, the need for an efficient, accurate and accessible tool for diagnosis is at a high demand. \gls{mpi} \gls{spect} provides critical information about the perfusion status of the heart, which helps in the early detection and planning the treatment which improves the outcomes of the patients.

Radionuclide \gls{mpi} under a specific condition, such as stress, is majorly regarded as one of the most effective diagnosis technique, which is also non-invasive, in order to identify the or detect the \gls{cad}. Using the application of \gls{mpi} \gls{spect}, clinicians become equipped to diagnose and detect the functionally relevant coronary stenoses with a relatively high level of specificity This actually enables them to make decisions that are informed and possibly the right ones regarding the pathways of the patients' treatment \cite{10.1001/jamacardio.2017.2471}. By visualizing the perfusion process of the heart muscles, clinicians can detect the areas of the heart where there is a presence of coronary stenoses or obstructions which may be the causing issue for inducible perfusion deficits under the conditions of stress or rest. This ability of diagnosis is not only essential to identify the patients with \gls{cad} but also functions as an important tool for mitigating patient risk and guiding the decision making process of the clinicians.

\gls{mpi} using \gls{spect} has emerged as both an effective and economically viable modality for the purpose of diagnosis. \gls{mpi} based \gls{spect} offers both the advantages of being accessible and having established standard clinical protocols hence it is the preferred choice of a number of diagnostic processes. One of the major strengths of \gls{mpi} is the adaptability of the technique, as it can incorporate a number of radio-pharmaceutical agents, such as 201Tl Chloride, 99mTc Tetrofosmin, and 99mTc Sestamibi, which basically is dependent upon the imaging protocols and imaging needs. The mentioned agents are typically administered intravenously before the image acquisition part, and then the collected image data are later reconstructed using techniques which are dedicatedly designed for cardiac imagery. The last, and possibly the most crucial, stages in the diagnostic process involves the segmentation of the anatomical structures relevant to the diseases and then the reorientation of this segmented volumetric data. This part of the diagnostic is usually performed by trained clinical professionals in order to ensure precision, better reliability and to mitigate the risks of errors.

Beyond the usage of the perfusion imaging alone, there are additional functional parameters, which are valuable, that can be derived when gated acquisition techniques are applied. These parameters include \gls{esv}, \gls{edv}, and the left ventricular ejection fraction (LVEF). All of the mentioned parameters are essential in order to indicate the performance of the heart. The values of these parameters are computed through the precise delineation of the myocardial boundaries of the \gls{lv},, which makes the task of segmentation even more crucial in the whole pipeline. The perfusion and the functional analysis collectively provide a detailed understanding of not only the vascular but also the mechanical health of the heart.

Efficient and accurate quantitative analysis of the 3D \gls{mpi} \gls{spect} data is extremely sensitive to a number of factors that are involved in the full end-to-end imaging and reconstruction pipeline, as mentioned above. All of these factors together contribute not only to the reliability of the evaluation of the data, but also to the detection of a range of cardiac abnormalities \cite{SLOMKA2012338}. The important step in this process is the segmentation and reorientation of the \gls{lv}, which basically refers to the determination of the spatial alignment of the \gls{lv} and its segmentation based on the anatomical midline. The tasks of both reorientation and the segmentation within \gls{mpi} \gls{spect} imaging have been acknowledged, for a long time, as one of the central yet difficult challenges. Over the course of years, multiple commercial systems have been developed in order to counter these issues, but more often relying on very extensive and curated datasets in order to ensure reliable performance in clinical environments \cite{Garcia2007}, \cite{Liu2007}, \cite{Ficaro2007}.However, the existing solutions fall short when they are applied to the newer reconstruction paradigms, especially in the situations where there are only a limited number of labled patients datasets. In order to mitigate these imitations faced by the current solutions and to increase the generalization capabilities of the models under limited data conditions, approaches incorporating self-supervised learning and few-shot learning have gained popularity. Nevertheless, the effectiveness of these strategies is most of the times overshadowed by the high costs associated with the expert annotations. In addition to this the lack of consensus regarding a standard segmentation protocol also complicate the practical application of the processes.

In the recent years, within the field of \gls{mpi} \gls{spect} imaging, the adoption of \gls{dl} techniques are looking at a significant revival \cite{tolu2025advancements}. This renewal is basically driven in part by the development of the novel radio-tracers and also the growing clinical demand to minimize the amount of administered radiation dose and also the image acquisition time of the performed procedures \cite{henzlova2011future}. As a consequence, the modern methods of reconstruction have been focusing on configurations that are based on low photon count data, sparse acquisition views and reduced amount of injected doses \cite{xie2023transformer}, \cite{xie2024generalizable}, \cite{chen2024dudocfnet}. But in-spite all that, the advancements do not fully resolve the challenges which are inherent to the segmentation tasks of \gls{mpi} \gls{spect}. Despite using \gls{sota} neural network based reconstruction strategies, the segmentation accuracy is still heavily relied on the underlying reconstructed images. When working with lower-dose inputs, the images mostly lack proper structural clarity, which diminishes the benefits which are offered by the \gls{dl} based reconstruction methods. Even in situations where the image reconstruction achieves are visual equivalence to a full-dose filtered back projection methods, the issues of low \gls{snr}, Poisson noise characteristics, and the impact of partial volume effect (PVE) continue to affect the generalization capabilities and hence the reliability of automated segmentation models.

In this work, is proposed a novel approach in order to solve the aforementioned bottlenecks of the segmentation task, all the while also contributing further detailed insights into the anatomical characteristics of the \gls{mpi} \gls{spect} \gls{lv}. Contrary to the previous approaches employed for the task, where the use of isolated pre-processed regions, or usage of cropped volumes, is common, the method in this study makes use of the entire reconstructed image volumes, hence incorporating all of the contextual spatial cues which are available within the full \gls{fov}. The choice of this design makes sure that no information that is diagnostically relevant is discarded, hence allowing the model to infer the left ventricular structure in relations to the surrounding regions of the anatomy. This holistic approach increases the robustness of the model, specifically in cases where abnormalities in the patterns could possibly interfere with the more localized analysis.

In order to mitigate the limitations hat are associated with \gls{cnn}, specifically their receptive field being restricted which hinders them from learning long-range dependencies, the proposed method employs a fully transformer based architecture called nnFormer \cite{zhou2023nnformer}. This architecture is specifically developed for tasks pertaining to volumetric medical imaging. It allows the network to learn global reasoning over the 3D structures which offers a significant advantage over the traditional \gls{cnn}s in situations where the boundaries of the organs are not sharply defined such as \gls{spect}. But there is a limitation to using transformer architectures, which is that they require a huge amount of data in order to learn acceptable global representations and have a good generalization ability. Hence, in order to overcome such a limitation, the proposed method incorporates \gls{ssp} as a regularization technique. Such shape priors introduce an anatomical consistency into the \gls{dl} model which acts as a guidance signal during the training of the model. This approach helps the models in situations where the amount of available data is limited. Using the shape priors, the model is made to learn meaningful and spatially coherent segmentation outputs even with minimal amount of supervision. This whole process bridges the gap between the traditional rule-based segmentation models and the fully data driven \gls{dl} approaches.
