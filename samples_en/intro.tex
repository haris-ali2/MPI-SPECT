\chapter{Introduction}
\label{ch:intro}

Myocardial Perfusion Imaging (MPI) using single-photon emission computed tomography (SPECT) pllays an important role in the process of non-invasive assessment of the coronary artery disease (CAD). Considerig cardiovascular diseases being one of the leading causes of mortality all across the world, the need for an efficient, accurate and accessible tool for diagnosis is at a high demand. MPI SPECT provides critial information about the perfusion status of the heart, which helps in the early detection and planning the treatment wich improves the outcomes of the patients.

Radionuclide MPI under a specific condition, such as stree, is majorly regarded as one of the most effective diagnosis technique, which ia also non-invasive, in order to identify the or detect the coronary artery disease (CAD). Using the application of MPI SPECT, clinicians become equipped to diagnose and detect the functionally relevant coronary stenoses with a relatively high level of specificity This actually enables them to decisions that are informed and possibly the right ones recarding the pathways of the patients' treatment \cite{10.1001/jamacardio.2017.2471}. One of the major strengths of MPI is the adaptability of the technique, as it can incorporate a number of radiopharmaceutical agents, such as 201Tl Chloride, 99mTc Tetrofosmin, and 99mTc Sestamibi, which basically is dependant upon the imaging protocols and imaging needs. The mentioned agents are typicallyy administered intravenously before the image acquisition part, and then the collected image data are later reconstructed using techniques which are dedicatedly designed for cardiac imagery. The last, and possibly the most crucial, stages in the diagnostic process involves the segmentation of the anatomical structures relevant to the diseases and then the reorientation of this segmented volumetric data. This part of the diagnostic is usually performed by trained clinical professionals in order to ensure precision, better reliability and to mitigate the risks of errors.

Efficient and accurate quantitative analysis of the 3D MPI SPECT data is extremely sensitive to a number of factors that are involved in the full end-to-end imaging and reconstruction pipeline. All of these factors together contribute not only to the reliability of the evaluation of the data, but also to the detection of a range of cardiac abnormalities \cite{SLOMKA2012338}. One important step in this process is the segmentation and reorientation of the LV, which basically refers to the determination of te spatial alignment of the LV and its segmentation based on the anatomical midline. The tasks of both reorientation and the segmentation within MPI SPECT imaging have been acknowledged, for a long time, as one of the central yet difficult challenges. Over the course of years, mutltiple commercial systems have been developed in order to counter these issues, but more often relying on very extensive and curated datasets in order to ensure reliable performance in clinical environments \cite{Garcia2007}, \cite{Liu2007}, \cite{Ficaro2007}.However, the eixisting solutions fall short when they are applied to the newer reconstruction paradigms, especially in the situations where there are only a limited number of labled patients datasets. In order to mitigate these imitations faced by the current solutions and to increase the generalization capabilities of the models under llimited data conditions, approaches incorporating self-supervised learning and few-shot learning have gained popularity. Nevertheless, the effectiveness of these strategies is most of the times overshadowed by the high costs associated with the expert annotations. In addition to this the lack of consensus regarding a standard segmentation protocol also complicate the practical application of the processes.

In the recent years, within the field of MPI SPECT imaging, the adoption of Deep Learning (DL) techniques are looking at a significant revival \cite{tolu2025advancements}. This renewal is basically driven in part by the development of the novel radiotracers and also the growing clinical demand to minimize the amount of administered radiation dose and also the image acquisition time of the performed procedures \cite{henzlova2011future}. As a consequence, the modern methods of reconstruction have been focusing on configurations that are based on low photon count data, sparse acquisition views and reduced amount of injected doses \cite{xie2023transformer}, \cite{xie2024generalizable}, \cite{chen2024dudocfnet}. But inspite all that, the advancements do not fully rosolve the challenges which are inherent to  the segmentation tasks of MPI SPECT. Despite using state-of-the-art neural network based reconstruction strategies, the segmentation accuracy is still heavily relied on the underlying reconstructed images. When working with lower-dose inputs, the images mostly lack proper structural clarity, which diminishes the benefits which are offered by the DL based reconstruction methods. Even in situations where the image reconstruction achieves are visual equivalence to a full-dose filtered back projection methods, the issues of low Signal-to-Noise Ratio (SNR), Poisson noise characteristics, and the impact of partial volume effect (PVE) continue to affect the generalization capabilities and hence the reliability of automated segmentation models.