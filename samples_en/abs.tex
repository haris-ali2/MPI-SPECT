\chapter*{Abstract}
\addcontentsline{toc}{chapter}{Abstract}

The manual delineating the \gls{lv} in \gls{mpi} is one of the most labor-intensive and time consuming tasks in nuclear cardiology and radiology. The outcome of the diagnosis of the \gls{mpi} is extremely dependent on the accuracy and the consistency of the segmentation of the ventricles, hence the process is done under extreme caution in order to minimize the risks of any possible error. However, the process of turning this task into an automated one present a number of challenges that need to be mitigated. Fist of all, the \gls{snr} is mostly low and the resolution of the image is limited, complicating the process of detecting the boundaries. Secondly, the high disparity in both the cardiac traces uptake ad the differences in the hardware used for the imaging introduces inconsistencies. Finally, their is a lack of a standardized definition of the shape of the \gls{lv} and the there is no standard shape that can be traced based purely on image data, which introduces a lot more ambiguity in the task.

This thesis proposes a novel method built to address the limitations mentioned above by using a Transformer-based architecture, integrating \gls{ssp} technique. This approach is specifically used to mitigate the data-hungry nature of the transformers in case of limited data. The proposed architecture achieves over 4\% improvement over a number of metrics in segmentation and classification against the bench-marked \gls{sota} approaches use for \gls{lv} segmentation, both on the synthetic data and the real-world clinical scans.

In addition to the improvements in the quantitative metrics, the incorporation of the prior shape information enabled the model to learn insights into the variability and the structural patterns of the \gls{lv} anatomy in \gls{mpi} \gls{spect} imaging. This deeper understanding of the \gls{lv} enhances the reliability of the AI-powered automatic segmentation of the \gls{lv} and also the general comprehension of the morphology of the \gls{lv} in clinical practice.